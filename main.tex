\documentclass[10pt, a4paper]{article} 
\usepackage{CJKutf8}
\usepackage{amsmath}
\usepackage[utf8]{inputenc}
\usepackage{multicol} %колонки
\usepackage{setspace} %межстрочный интервал
\usepackage{ragged2e}% выравнивания текста по ширине в документе.
\usepackage{fancyhdr} %настройки верхнего и нижнего колонтитулов в документе.
\usepackage{titlesec} %стилей заголовков разделов в документе.
\usepackage{enumitem} %настройки списков в документе.
\usepackage {graphicx}%Вставка картинок 
\usepackage[centercaption]{sidecap}
\usepackage{subfigure}
\usepackage{float}%"Плавающие" картинки
\usepackage{wrapfig}%Обтекание фигур (таблиц, картинок и прочего)

\usepackage[left=1.9cm,right=1.9cm, top=2.2cm,bottom=2.5cm]{geometry}
\justifying % выравнивает текст по ширине.
\fancyhf{} %очищает все верхние и нижние колонтитулы.
\renewcommand{\headrulewidth}{0pt} % remove the header rule
\cfoot{\vskip -1.5cm \thepage} %устанавливает номер страницы в нижнем колонтитуле.
\linespread{0.84} %устанавливает межстрочный интервал 
\setlength{\columnsep}{0.5cm}
\setcounter{page}{36}
\renewcommand{\thesection}{\Roman{section}} %устанавливает стиль нумерации разделов в виде заглавных римских цифр.
\titleformat{\section}{\footnotesize\centering\sc}{\thesection.}{0cm}{}[] %настраивает стиль заголовков разделов.
\begin{document}
\fontsize{10}{14}\selectfont
\begin{multicols}{2}
\setlength{\parindent}{0.8cm}
\setlength{\parindent}{0.4cm}
\fontsize{10}{15}\selectfont
\par\ The features for a finite set can be computed as a
generalized or stepped average:
\begin{center}
 $\sqrt[p]{\frac{\sum_{x\in\mathbf{S^{{x}^{p}}}}}{\mid{\mathbf{S}\mid} }}$\qquad      
\end{center} 
\par\  For a finite set of objects (an element of the product
of metric spaces), metrics based on the (stepped) mean
can be considered
\[d(\langle P,Q \rangle)=\sqrt[p]{\sum_ { x \in\mathbf{P}}\sum_ { y \in\mathbf{Q}}d(\langle x,y \rangle)^p }\]
\par\  The transition from an unbounded metric to a metric
as a bounded monotone (fuzzy) measure, can be realized
according to:
\[\lim_ { z \to\ d(\langle x,y \rangle) }
    \frac{z}{1+z}\]
or any other semi-additive monotonically increasing
bounded function from the point (0; 0):
\begin{center}
    $f(x+y)\le f(x)+f(y)$
\end{center}
\par\ For a metric on linear representations by generalized
strings, we can use either the edit distance between them
or the minimum of the edit distances between the results
of the action of a subgroup of the symmetric group
of permutations of linear representations by generalized
strings.
\par \ Let us consider an algorithm for finding the extensional closure of a concept:
\begin{itemize} [noitemsep] \fontsize{10}{15}\selectfont
    \item[1] The distensible set of the extensional closure is empty.
    \item[2] Include a concept in the current front.
    \item[3] Construct a new next front by going deep down the extensional.
    \item[4]  Subtract the current front from the next front.
    \item[5] Add the elements of the current edge to the distensional set of the extensional closure.
    \item[6] If the next edge is empty return the extensional closure.
    \item[7] Make the next edge the current edge, go to step 3.
    
\end{itemize} 
\linespread{0.8}
\par \ Consider an algorithm for computing the metric
(quasi-metric) on the union of (finite) extensional closures of concepts:

\begin{itemize}[noitemsep] \fontsize{10}{15}\selectfont
 \item[1] Find the extensional closure of the first concept.
 \item[2] Find the extensional closure of the second concept.
 \item[3] If the intersection of the found closures is empty
then return the metric equal to infinity.
\item[4] If one concept lies in the extensional closure of
another, then find the distance from the other to
the first and return the metric equal to the smallest
of the found ones.
\item[5] Descend from each concept deep into the extensional closure to the intersection of the extensional
closures and remember the lengths of descent as \textit{A} 
and \textit{B}.
\item[6] Stepwise descent in the intersection of closures to continue until the moment of meeting, calculate the
corresponding lengths \begin{math} A'
\end{math} and \begin{math} B'.
\end{math}
\item[7] Return the metric  \begin{math} A + A' + (B+B') * 1. \end{math}
\end{itemize}
\par \ Each static structure (with its own denotational semantics [19]) and its characteristics can be related (see Fig.6, Fig.7) to the dynamic structure of a (formal) information processing model based on that static structure and its operations (possibly within \begin{math} {A^{(+_{7}^{*})}}
\end{math} [57]. Such dynamic structures and their corresponding operations are described by a big-step operational semantics. The connection between small-step [16], [52] and big-step operational semantics [6] can be revealed through the connections between the operational-information space and the (formal) information processing model. Operational information space can be described as follows. A set of data instances \textit{V}, a set of channels \textit{P}, a set of operators \textit{K}, a set of (commutative) operations \textit{O}, a set of configurations \textit{C}, a sequence of configurations \textit{R} 
\begin{center}
\begin{math} R \subseteq C × C
\end{math}
\end{center}
\begin{center}
\begin{math}
O \subseteq {2^{V^{2}×V}} \cup {2^{V×V^{2}}}
\end{math} 
\end{center}
\[C \subseteq {2^{K\cup P\cup (P×(V\cup K))\cup (K×(O\cup P)) }}\]
\linespread{0.8}
If a new parameter (with a value) is added to a configuration operation, it is possible to jump to the updated configuration that contains it.
\par \ If a new operation is added to the configuration parameters, it is possible to switch to the updated configuration that contains it.
\par \ It is possible to switch to a configuration that does not contain an operation.
\par \ It is possible to jump to a configuration that does not contain a parameter that is not used by an operation.
\par \ Consider the flows of an open (acyclic (Fig.2)) or closed path (cyclic (Fig.3)) for the corresponding open or closed structures and assume the following requirements for its flow $c_{ij}$ [57].
\par \ Each edge is mapped to a flow (energy) $c_{ij}$. To each
vertex s there is a flow \begin{math} c_s=\sum{{^{n}_{j=1}}}\ c_{sj}\end{math}. 
In addition to the forward flow, the reverse flow \begin{math}
 c{{^{-1}_{ji}}}, c{{^{-1}_{s}}}=\sum{{^{n}_{j=1}}} c{{^{-1}_{sj}}} 
\end{math} is also computed. Their difference is equal to:\begin{math}\
d_{ij} = c_{ij} - c{{^{-1}_{ji}}}, d_s=c_s-c{{^{-1}_{s}}}.
\end{math}\ The direct (local) amplitude is
calculated as follows \begin{math} p_{ij}=\frac{d_{ij}}{c_i+c{{^{-1}_{i}}}} + \frac{1}{\sum{{^{V}_{j=1}}}\ a_{ij}}.
\end{math}

\[\sum_{j=1}^{N}c_{ij}=\sum_{j=1}^{N}c_{ji}\]

\par
  \[c_{ij}=\frac{\sum{{^{N}_{j=1}}}{c_{ij}}}{{\sum{{^{N}_{j=1}}}{a_{ij}}}}* a_{ij};c_{ij}*\sum_{j=1}^{N}a_{ij}=a_{ij}*\sum_{j=1}^{N}c_{ij}\]
Also in matrix form we have:
\[A^{T}*C=(A*1)\bullet C\]
 For the structure in Fig.3 and its forward flow we have:
 \begin{equation*}
 \begin{cases}
   c_{11} = c_{12}
   \\
  c_{56}=c_{57}
   \\
   c_{89}=c_{814}
   \\
   c_{1112}=c_{1113}
 \end{cases}
\end{equation*}
\ Let us find the minimum solution in natural numbers.
\par As a result of fulfillment of these requirements we
obtain the following table of results (see Fig.3, Fig.4,
Fig.8, Fig.5).
%таблица
\begin{center}
 \caption {\footnotesize{Table I \par 
Table of dynamic structure characteristics}\vspace{0.3cm}}
\resizebox{0.35\textwidth}{!}{

\begin{tabular}{ |c|c|c|c|}

\hline
\textbf {Edge} & \textbf{Flow} & \textbf {Forward} & \textbf{Backward} \\ \textbf {number} & \textbf{difference} & \textbf {amplitude } & \textbf{amplitude} \\ 
\hline
0 & 3 & 19/32 & 35/35=1  \\
\hline
1 & -3 & 13/32 & 29/29=1 \\ 
     \hline
2 & 3 & 35/35=1 & 35/35=1 \\
 \hline
 3 & -3 & 29/29=1 & 29/29=1\\
 \hline
4 & 3 & 35/35=1 & 19/34\\
\hline
5 & -1 & 15/29 & 15/34\\
\hline
6 & -2 & 14/29 & 30/30=1\\
\hline
7 & 2 & 34/34=1 & 34/34=1 \\
\hline
8 & -2 & 30/30=1 & 30/30=1 \\
\hline
9 & 0 & 16/34=8/17 & 16/30=8/15\\
\hline
10 & 2 & 18/34=9/17 & 18/35\\
\hline
11 & -2 & 30/30=1 & 14/30=7/15\\
\hline
12 & -2 & 30/30=1 & 30/30=1\\
\hline
13 & 1 & 17/30 & 33/33=1\\
\hline
14 & -3 & 13/30 & 29/29=1 \\
\hline
15 & 1 & 33/33=1 & 17/35\\
\hline
16 & -3 & 29/29=1 & 29/29=1\\
\hline
17 & 3 & 35/35=1 & 35/35=1\\
\hline
18 & -3 & 29/29=1 & 13/32\\
\hline
19 & 3 & 35/35=1 & 19/32 \\
\hline
 \end{tabular} }
 \end{center}
\par \ An analogous result can be obtained for an open (nonclosed) structure (Fig. 2).
\par \ Each strongly connected structure has an (own) period \textit T, which is the GCD of all periods (lengths of simple 
cycles) in this structure, and has a partition by levels of
wave fronts corresponding to this period. We will call the
number of these levels the length of the structure \textit {L = T}.
The length \textit L of an acyclic structure is the maximum
length of the shortest route for two connected vertices.
Each (acyclic) structure has a mapping \textit W of the set of
numbers of moments of time [9] to the set of subsets
of vertices by levels of wavefronts at given moments of
time, the number of which does not exceed the length
and diameter of the structure. Each wavefront has an
energy $E(t)=\sum_{s\in W(t)}c_s.$  The wavefront energy can be different from 1. The amplitude at the top of the
wavefront $p_s^t=\frac{c_s}{E(t)}$ is in the interval [0; 1]. The average amplitude is inversely proportional to the number of front elements $\frac{E(t)}{|W(t)|}$.
\par \ The amount of information of the wavefront at the
moment \textit{t} is expressed.
\[ -\sum_{s\in W(t)} \Bigg(\frac{|U_s^{(t)}|}{p_s^t}*\ln\frac{|U_s^{(t)}|}{p_s^t}\Bigg)\]
\vspace{-0,7cm}
\setcounter{figure}{1}
\begin{figure}[H]
\centering
\includegraphics[width=1.1\linewidth]{1.png}
\caption{\small Acyclic orgraph weighted structure.}
\label{fig:mpr}
\end{figure}
 \vspace{-0,5cm}
\begin{figure}[H]
\centering
\includegraphics[width=1.1\linewidth]{2.png}
\caption{\small Strong connected flow difference weighted orgraph.}
\label{fig:mpr}
\end{figure}
Set of undistinguishable vertices of the wavefront
     \[\left\{s\right\} \subseteq U_s^{(t)}\subseteq W(t)\]  
\ The average (arithmetic) amount of information of the
structure:
\[ -\frac{1}{T}*\sum_{t=1}^T \sum_{s\in W(t)} \Bigg(\frac{|U_s^{(t)}|}{p_s^t}*\ln\frac{|I_{(t)}|*{|U_s^{(t)}|}}{T*{p_s^t}}\Bigg)\]
 \vspace{-0,79cm}
\begin{figure}[H]
\centering
\includegraphics[width=1.1\linewidth]{3.png}
\caption{\small Strong connected orgraph forward flow.}
\label{fig:mpr}
\end{figure}
\end{multicols}
\begin{figure}[H]
\centering
\includegraphics[width=0.8\linewidth]{4.png}
\caption{\small  Strong connected orgraph flow difference weights matrix.}
\label{fig:mpr}
\end{figure}
\begin{figure}[H]
\centering
\includegraphics[width=0.85\linewidth]{5.png}
\caption{\small Asymmetric ontological structure with corresponding dynamic structure.}
\label{fig:mpr}
\end{figure}
\begin{multicols}{2}
\setlength{\parindent}{0.8cm}
\setlength{\parindent}{0.4cm}
\fontsize{10}{15}\selectfont
\par \ A set of indistinguishable moments in time
\begin{center}
\begin{math} \{t\} \subseteq I_{(t)}  \subseteq Dom(W)
\end{math}
\end{center}
\par \ The information in a strongly connected structure will be called real (elliptic), and in an acyclic structure will be called imaginary (hyperbolic).
\par \ Two kinds of structures are considered: a (finite)
acyclic graph and a strongly connected pseudograph. An
arbitrary (finite) pseudograph structure can be decomposed into its connected components. An arbitrary (finite) connected pseudograph structure can have different
kinds of substructures and, in particular, can be broken
down into the two kinds of structures discussed earlier: strongly connected components (subpseudographs),
acyclic graphs (subgraphs). However, within a structure,
these substructures can have different relationships and
fulfill different roles [66]. Let us describe different types of substructures according to their roles (relations) performed (available) in the structure [66].
\par \ Resonators are maximal strongly connected subpseuwhose elements at least one other (different) 
\end{multicols}
\end{document}